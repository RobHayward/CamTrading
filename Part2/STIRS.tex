\documentclass[14pt,xcolor=pdftex,dvipsnames,table]{beamer}\usepackage[]{graphicx}\usepackage[]{color}
%% maxwidth is the original width if it is less than linewidth
%% otherwise use linewidth (to make sure the graphics do not exceed the margin)
\makeatletter
\def\maxwidth{ %
  \ifdim\Gin@nat@width>\linewidth
    \linewidth
  \else
    \Gin@nat@width
  \fi
}
\makeatother

\definecolor{fgcolor}{rgb}{0.345, 0.345, 0.345}
\newcommand{\hlnum}[1]{\textcolor[rgb]{0.686,0.059,0.569}{#1}}%
\newcommand{\hlstr}[1]{\textcolor[rgb]{0.192,0.494,0.8}{#1}}%
\newcommand{\hlcom}[1]{\textcolor[rgb]{0.678,0.584,0.686}{\textit{#1}}}%
\newcommand{\hlopt}[1]{\textcolor[rgb]{0,0,0}{#1}}%
\newcommand{\hlstd}[1]{\textcolor[rgb]{0.345,0.345,0.345}{#1}}%
\newcommand{\hlkwa}[1]{\textcolor[rgb]{0.161,0.373,0.58}{\textbf{#1}}}%
\newcommand{\hlkwb}[1]{\textcolor[rgb]{0.69,0.353,0.396}{#1}}%
\newcommand{\hlkwc}[1]{\textcolor[rgb]{0.333,0.667,0.333}{#1}}%
\newcommand{\hlkwd}[1]{\textcolor[rgb]{0.737,0.353,0.396}{\textbf{#1}}}%
\let\hlipl\hlkwb

\usepackage{framed}
\makeatletter
\newenvironment{kframe}{%
 \def\at@end@of@kframe{}%
 \ifinner\ifhmode%
  \def\at@end@of@kframe{\end{minipage}}%
  \begin{minipage}{\columnwidth}%
 \fi\fi%
 \def\FrameCommand##1{\hskip\@totalleftmargin \hskip-\fboxsep
 \colorbox{shadecolor}{##1}\hskip-\fboxsep
     % There is no \\@totalrightmargin, so:
     \hskip-\linewidth \hskip-\@totalleftmargin \hskip\columnwidth}%
 \MakeFramed {\advance\hsize-\width
   \@totalleftmargin\z@ \linewidth\hsize
   \@setminipage}}%
 {\par\unskip\endMakeFramed%
 \at@end@of@kframe}
\makeatother

\definecolor{shadecolor}{rgb}{.97, .97, .97}
\definecolor{messagecolor}{rgb}{0, 0, 0}
\definecolor{warningcolor}{rgb}{1, 0, 1}
\definecolor{errorcolor}{rgb}{1, 0, 0}
\newenvironment{knitrout}{}{} % an empty environment to be redefined in TeX

\usepackage{alltt}

% Specify theme
\usetheme{Madrid}
% See deic.uab.es/~iblanes/beamer_gallery/index_by_theme.html for other themes
\usepackage{caption}
\usepackage[comma, sort&compress]{natbib}
\usepackage{tikz}
\usepackage{graphicx}
\usetikzlibrary{arrows,positioning}
\graphicspath{{../Pictures/}}
\usepackage{amsmath}
\bibliographystyle{agsm}
% Specify base color
\usecolortheme[named=OliveGreen]{structure}
% See http://goo.gl/p0Phn for other colors

% Specify other colors and options as required
\setbeamercolor{alerted text}{fg=Maroon}
\setbeamertemplate{items}[square]

\AtBeginSection[]{
\begin{frame}
\vfill
\centering
\begin{beamercolorbox}[sep=8pt, center, shadow=true, rounded=true]{title}
\usebeamerfont{title}\insertsectionhead\par%
\end{beamercolorbox}
\vfill
\end{frame}
}

% Title and author information
\title{STIRS and interest rate expectations}
\author{Rob Hayward}
%\date{Feb 2018}
\IfFileExists{upquote.sty}{\usepackage{upquote}}{}
\begin{document}

\begin{frame}
\titlepage
\end{frame}

\begin{frame}{Outline}
\tableofcontents
\end{frame}

\section{Introduction}
\begin{frame}{Expectations}
However, expectations are formed, 
\begin{itemize}[<+-| alert@+>]
\pause
\item Financial markets are moved by \emph{surprises}
\item This makes it imperative to know what is embedded in the current market price
\item There is some sort of distribution of expectations with a range and an expected (mean) value
\item Shackle and \emph{Kalediostatics} 
\end{itemize}
\end{frame}

\begin{frame}{Interest rate expectations}
There are a number of ways that interest rate expectations can be determined 
\begin{itemize}[<+-| alert@+>]
\pause
\item A poll of people (economists?)
\item Short-term interest rate futures (STIRS)
\item Implied forward rates
\end{itemize}
\end{frame}

\section{US interest rates}

\begin{frame}{US interest rates}
\begin{knitrout}
\definecolor{shadecolor}{rgb}{0.969, 0.969, 0.969}\color{fgcolor}
\includegraphics[width=\maxwidth]{figure/money-1} 

\end{knitrout}
\end{frame}

\section{Specifications}
\begin{frame}{CME Eurodollar}
\begin{itemize}[<+-| alert@+>]
\pause
\item Unit of trading: \$1000000
\item Delivery: Mar, Jun, Sep and Dec. 
\item Quotation: 100 minus rate of interest
\item Minimum price fluctuation: 0.005 (0.0025 first month)
\item $ \$1000000 \times 0.05/100 \times 0.25 = \$12.50$
\item Last trading day: 10.00 two business days prior to the third Wednesday of the delivery month
\end{itemize}
\end{frame}

\begin{frame}{Specification}
\begin{center}
\rowcolors{1}{OliveGreen!20}{OliveGreen!5}
\begin{tabular}{llll}
Contract & Code & Price & Value\\
\hline
Mar 2018 & H18 & 98.155& 1.845\\
Jun 2018 & M18 & 97.955 & 2.045\\ 
Sep 2018 & U18 & 97.82 &2.18\\
Dec 2018 & Z18 & 97.70& 2.30\\
\hline
\end{tabular}
\end{center}
\href{www.cmegroup.com/trading/interest-rates/eurodollar.html}{Latest CME Prices}
\end{frame}

\begin{frame}{Trading opportunities}
\begin{itemize}[<+-| alert@+>]
\pause
\item Rate expectations
\item Natural levels for expectations (support and resistanc)
\item Spreads between different months (reduces risk)
\item Spreads between different counties (link to foreign exchange)
\end{itemize}
\end{frame}

\section{Further reading}
\begin{frame}{Further reading}
\begin{itemize}[<+-| alert@+>]
\pause
\item \href{http://timharford.com/2016/03/how-to-make-good-guesses/}{Tim Harford}
\item \href{www.cmegroup.com/trading/interest-rates/eurodollar.html}{Latest CME Prices}
\item \href{https://www.theice.com/products/38527986/Three-Month-Euribor-Futures}{ICE 3m Euribor}
\item \href{https://www.marxists.org/reference/subject/economics/keynes/general-theory/ch12.htm}{Chapter 12: The State Of Long-Term Expectations (Keynes)} 
\end{itemize}
\end{frame}

\begin{frame}{Open outcry}
\begin{center}
\includegraphics[height = 2.6in, trim = 00 20 00 20]{Open-Out-Cry.jpg}
% trim is left, lower, right, upper
\end{center}
\href{https://www.youtube.com/watch?v=RLySXTIBS3c}{Trading Places}
\end{frame}


\end{document}
